The breast cancer dataset was the second mandatory dataset.
We got a score of 97.647\% on kaggle. 
The dataset is is used to classify wheather a tumour is harmful or not.

\subsection{Characteristics}

\begin{itemize}
\item No missing values
\item Binary target class (B and M)
\item Only rational features
\item 30 attributes (disregarding id)
\item 285 samples
\end{itemize}

The attributes contain descriptions of the tumour, for example its radius.

\subsection{Characteristics of Target Value}

A tumour is a cluster of abnormal cells.
If it is a benign tumour it does not contain cancerous cells, whereas a malignant tumour does hold cancerous cells.
Therefore it is more important to correctly classify a sample if it is a malignant tumour than correctly classifying a benign tumour.
It contains two target classes, Benign (B) and Malignant (M).
While 96 of the samples are of type M, 189 are of type B.

\image{breast/plots/countplot.png}{Histogram of the target values}{\label{fig:breast-target}}

% \begin{figure}[H]
%   \begin{center}
%     \includegraphics[width=0.8\linewidth]{breast/plots/countplot.png}
%     \caption{Histogram of the target values}
%     \label{fig:breast-target}
%   \end{center}
% \end{figure}

\subsection{Feature Selection}
First, different plots to visualize the data were made.
An example can be seen in Figure \ref{fig:breast-violin}, where it can be observed that the fractalDimensionMean (the feature on the right) is bad for classification while the radiusMean could be more useful as its classes are differently distributed and the mean differs significantly.
Additionally, boxplots were made, which can be observed in Figure \ref{fig:breast-boxplot}.
Boxplots give information about outliers, but the distribution can not be observed as clearly as with violin plots.
Often attributes contain the same information, which can be checked by calculating the correlation of two attributes or graphically interpret their scatter plot as shown in Figure \ref{fig:breast-correlation}.
These plots were analysed and the findings were used afterwards to select the best attributes.

\image{breast/plots/violinplot.png}{Violine plot of the first 10 features that were scaled by MinMax}{\label{fig:breast-violin}}
\image{breast/plots/boxplot.png}{Boxplot of the first 10 features that were scaled by MinMax}{\label{fig:breast-boxplot}}
\image{breast/plots/corr_comparision.png}{Scatter plot of two features from the breast cancer dataset}{\label{fig:breast-correlation}}

% \begin{figure}[H]
%   \begin{center}
%     \includegraphics[width=0.8\linewidth]{breast/plots/violinplot.png}
%     \caption{Histogram of the target values}
%     \label{fig:breast-violin}
%   \end{center}
% \end{figure}

Furthermore, recursive feature selection with the Random Forest Classifier was performed as well, which can be observed in Figure \ref{fig:breast-feature-selection}. The results showed that 15 to 30 features work best, which confirms later findings, that feature selection does not necessarily improve performance, since there are only 30 features to select.

\image{breast/plots/rf_feature_selection.png}{Recursive Feature selection}{\label{fig:breast-feature-selection}}

\subsection{K Nearest Neighbors Classifier}

For the K Nearest Neighbour Classifier a grid search with cross validation was used to test different \textit{k}'s from 1 to 30 and different metrics, such as the Euclidean, Chebyshev and Manhattan distance.

As seen in Figure \ref{fig:breast-knn-metrics} Euclidean works best for preprocessed data, whereas Manhattan is better for non preprocessed data.
Concerning the amount of neighbours included in the majority vote, preprocessing works best with a \textit{k} of 10 and weighted distance and non preprocessed with a \textit{k} of 5 and uniform weighted distance.
Comparing the performance of two estimators, preprocessing had a huge impact, which resulted in a 4\% higher accuracy, archiving a performance of 98.6\%.  

Interestingly, Chebychev as a distance metric performs significantly worse compared to the others, especially with a higher \textit{k}.

\image{breast/plots/knn_p_comparision.png}{Comparison of metrics}{\label{fig:breast-knn-metrics}}
% \begin{figure}[H]
%   \begin{center}
%     \includegraphics[width=0.8\linewidth]{breast/plots/knn_p_comparision.png}
%     \caption{Histogram of the target values}
%     \label{fig:breast-knn-metrics}
%   \end{center}
% \end{figure}

The previously explained feature selection unfortunately made no improvement, it made the performance even worse, which is shown in the Figure \ref{fig:breast-knn-comparison}.

\image{breast/plots/knn_feature_comparision.png}{Comparison of all Features and selected Features}{\label{fig:breast-knn-comparison}}
% \begin{figure}[H]
%   \begin{center}
%     \includegraphics[width=0.8\linewidth]{breast/plots/knn_feature_comparision.png}
%     \caption{Histogram of the target values}
%     \label{fig:breast-knn-comparison}
%   \end{center}
% \end{figure}

\subsection{Random Forest Classifier}

First a randomized search with cross validation was performed to get a first impression for good parameters for the random forest classifier.
It was discovered that the \textit{min\_sample\_split} works best if it is set to 0.04, e.g. the fraction of the minimum samples required to split an internal node is 0.04.
Using the previously obtained information about the parameters, a grid search with cross validation was performed.
It was found that a score of 97.89\% can be obtained by using a maximum of 10\% of the features for an estimator and 47 estimators in total.

\image{breast/plots/rf_np_comparision.png}{Comparison of Random Forest by estimator count and max features}{\label{fig:breast-rf-comparison}}

\subsection{Multi-Layer Perceptron Classifier}

For the Multi-Layer Perceptron Classifier a grid search was performed.
Different layer architectures and activation functions were altered.
As seen in Figure \ref{fig:breast-mlp-comparison}, the best results were obtained with the activation function relu and 3 hidden layers with 30, 15 and 30 neurons.
However the results are mostly the same. However  this is not the case for the Sigmoid or logistic activation function, which is probably due to vanishing gradients.

\image{breast/plots/mlp_p_comparision.png}{comparison of layer sizes and activation functions}{\label{fig:breast-mlp-comparison}}

Preprocessing improved the result as shown in Figure \ref{fig:breast-mlp-np-p-comparison}.
Interestingly the gradients did not vanish for the logistic activation function without preprocessing.

\image{breast/plots/mlp_np_p_comparision.png}{Comparison of activation functions with preprocessing in orange and without preprocessing in blue}{\label{fig:breast-mlp-np-p-comparison}}

\subsection{Conclusion}

Preprocessing leads to better results for all classifiers.
However feature selection leads to worse results.
Therefore the data should only be scaled and all features should be used for classification. Surprisingly, for the breast cancer dataset, K Nearest Neighbours performed better than the MLP in terms of accuracy.

\begin{table}[H]
\begin{center}
\begin{tabular}{|l|l|l|}
\hline
                       & Preprocessing & No-Preprocessing \\ \hline
KNeighborsClassifier   & 0.9862        & 0.9474           \\ \hline
RandomForestClassifier & 0.9789        & 0.9789           \\ \hline
MLPClassifier          & 0.9824        & 0.9614           \\ \hline
\end{tabular}
\caption{Comparison of accuracy of different techniques with- and without preprocessing}
\end{center}
\end{table}

Holdout and cross validation yield similar results.
Therefore holdout can be used for parameter evaluation.

\begin{table}[H]
\begin{center}
\begin{tabular}{|l|l|l|}
\hline
                       & Holdout & Cross Validation \\ \hline
KNeighborsClassifier   & 0.9824  & 0.9862           \\ \hline
RandomForestClassifier & 0.9824  & 0.9789           \\ \hline
MLPClassifier          & 0.9824  & 0.9858           \\ \hline
\end{tabular}
\caption{Comparison of accuracy of holdout versus cross-validation}
\end{center}
\end{table}

The best result was obtained by the K Nearest Neighbours classifier with 98.62\% and the second best by the MLP classifier with 98.58\%.
However on kaggle, KNN only archived a score of 95.29\%, whereas the MLP archived a score of 97.64\%. The reason behind this might be the fact, that on kaggle more training data can be used and this can change the results significantly.
Regarding the runtime KNN classifier is the fastest, followed by the random forest classifier and then the multi layer perceptron, which takes the longest to train.

Concerning the classification of a cancerous tumour cells, it is not an easy task, since wrong prediction can be fatal for the patient.

Therefore, the correct prediction of a malignant as a non cancerous tumour is not that important as the classification of a benign cell. Thus, the best performance metric is the one that minimizes false negatives, which is the case, when using Recall as a performance metric, which also means that KNN is the best classifier in our case for this task, however more samples should be used for a better evaluation, hence a score of 98.58\% is still not sufficient.

\begin{table}[H]
\begin{center}
\begin{tabular}{|l|l|l|l|l|l|}
\hline
                       & Accuracy & Precision & Recall & F1     & Runtime (sec) \\ \hline
KNeighborsClassifier   & 0.9862   & 0.9902    & 0.9800 & 0.9841 & 0.0016        \\ \hline
RandomForestClassifier & 0.9756   & 0.9752    & 0.9715 & 0.9726 & 0.0677        \\ \hline
MLPClassifier          & 0.9858   & 0.9902    & 0.9783 & 0.9831 & 0.8999        \\ \hline
\end{tabular}
\caption{Comparison of different performance metrics and runtimes}
\end{center}
\end{table}

